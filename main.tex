%%------------------------------------------------%%
%%  Copyleft by Karol Stepanienko                 %%
%%  Based on https://github.com/ArturB/WUT-Thesis %%
%%                 March 2022                     %%
%%------------------------------------------------%%

\documentclass[
    bindingoffset=5mm,  % Binding offset
    footnoteindent=3mm, % Footnote indent
    hyphenation=true    % Hyphenation turn on/off
]{src/wut-thesis}

\graphicspath{{tex/img/}} % Katalog z obrazkami.
% \addbibresource{bibliografia.bib} % Plik .bib z bibliografią
%-------------------------------------------------------------
% Wybór wydziału:
%  \facultyeiti: Wydział Elektroniki i Technik Informacyjnych
%  \facultymeil: Wydział Mechaniczny Energetyki i Lotnictwa
% --
% Rodzaj pracy: \EngineerThesis, \MasterThesis
% --
% Wybór języka: \langpol, \langeng
%-------------------------------------------------------------
\facultyeiti    % Wydział Elektroniki i Technik Informacyjnych
\langeng % Praca w języku angielskim


%--------------
% Zmiana nazwy podpisów pod rysunkami
%--------------
\renewcommand{\figurename}{Zrzut ekranu}
\addto\captionspolish{\renewcommand{\figurename}{Zrzut ekranu}}
%\def\graphname{Wykres}
%\def\tablename{Tab.}

\begin{document}
%------------------
% Strona tytułowa
%------------------
\title{
        Report template based on~thesis template
}
\author{Name Surname}
\album{123456}
\date{\the\year}
\maketitle

%--------------
% Spis treści
%--------------
% \cleardoublepage % Zaczynamy od nieparzystej strony
\clearpage
\tableofcontents

%------------
% Rozdziały
%------------
% \cleardoublepage % Zaczynamy od nieparzystej strony
\clearpage
\pagestyle{headings}

% Wygodnie jest trzymać każdy rozdział w osobnym pliku.
% Umożliwia to również łatwą migrację do nowej wersji szablonu:
% zazwyczaj wystarczy podmienić plik src/wut-thesis.cls
%\input{tex/1-cos}
%%------------------------------------------------%%
%%  Copyleft by Karol Stepanienko                 %%
%%  Based on https://github.com/ArturB/WUT-Thesis %%
%%                 March 2022                     %%
%%------------------------------------------------%%

\section{Rozdział 1}

\subsection{Podpunkt 1}
\begin{figure}[th]
	\centering
	\includegraphics{tex/img/stock.jpg}
	\caption{Podpis}
	\label{stock}
\end{figure}
\clearpage

\subsection{Podpunkt 2}
Lorem ipsum dolor sit amet, consectetur adipiscing elit, sed do eiusmod tempor incididunt ut labore et dolore magna aliqua. Ut enim ad minim veniam, quis nostrud exercitation ullamco laboris nisi ut aliquip ex ea commodo consequat. Duis aute irure dolor in reprehenderit in voluptate velit esse cillum dolore eu fugiat nulla pariatur. Excepteur sint occaecat cupidatat non proident, sunt in culpa qui officia deserunt mollit anim id est laborum.
\clearpage

\subsection{Podpunkt 3}
Lorem ipsum dolor sit amet, consectetur adipiscing elit, sed do eiusmod tempor incididunt ut labore et dolore magna aliqua. Ut enim ad minim veniam, quis nostrud exercitation ullamco laboris nisi ut aliquip ex ea commodo consequat. Duis aute irure dolor in reprehenderit in voluptate velit esse cillum dolore eu fugiat nulla pariatur. Excepteur sint occaecat cupidatat non proident, sunt in culpa qui officia deserunt mollit anim id est laborum.
\clearpage


 % Można też pisać rozdziały w jednym pliku.
\clearpage % Zawsze zaczynamy rozdział od nowej strony

%---------------
% Bibliografia
%---------------
%\cleardoublepage % Zaczynamy od nieparzystej strony
%\printbibliography

%--------------------------------------
% Spisy: rysunków, tabel, załączników
%--------------------------------------
\clearpage
\pagestyle{plain}

\listoffigurestoc    % Spis rysunków.
%\vspace{1cm}         % vertical space
%\listoftablestoc     % Spis tabel.
%\vspace{1cm}         % vertical space
%\listofappendicestoc % Spis załączników

% Wykaz symboli i skrótów.
% Pamiętaj, żeby posortować symbole alfabetycznie
% we własnym zakresie. Makro \acronymlist
% generuje właściwy tytuł sekcji, w zależności od języka.
% Makro \acronym dodaje skrót/symbol do listy,
% zapewniając podstawowe formatowanie.
%\vspace{0.8cm}
%\acronymlist
%\acronym{EiTI}{Wydział Elektroniki i Technik Informacyjnych}
%\acronym{PW}{Politechnika Warszawska}

%-------------
% Załączniki
%-------------

% Obrazki i tabele w załącznikach nie trafiają do spisów
%\captionsetup[figure]{list=no}
%\captionsetup[table]{list=no}

% Używając powyższych spisów jako szablonu,
% możesz dodać również swój własny wykaz,
% np. spis algorytmów.

\end{document} % Dobranoc.
